%%%%%%%%%%%%%%%%%%%%%%%%%%%%%%%%%%%%%%%%%%%%%%
%% Exemple of a report
%%
%% Vilsafur
%%
%% v1   - 04/2016 :
%%
%% TODO :
%%%%%%%%%%%%%%%%%%%%%%%%%%%%%%%%%%%%%%%%%%%%%%
\documentclass[french]{VGReport}


%%%%%%%%%%%%%%%%%%%%%%%%%%%%%%%%%%%%%%%%%%%%%%%%%%%
%% GENERAL INFORMATION
%%%%%%%%%%%%%%%%%%%%%%%%%%%%%%%%%%%%%%%%%%%%%%%%%%%
%TODO Title of document
\title{The title}

%TODO authors
\author{
    Prenom1 Nom1,
    Prenom2 Nom2 \&
    Prenom3 Nom3
}

\begin{document}

% création de la page de titre
\maketitle

% création de la table des matières
\MyToc

% Justification moins stricte : des mots ne dépasseront pas des paragraphes
\sloppy

\section{first section}
\subsection{first subsection}
\label{sec:first_subsection}

\begin{enumerate}
    % Comme pour itemize, chaque élément est défini avec la commande '\item'
    \item premier élément ;
    \item 2 élément ;
    \item troisième élément.
    \begin{enumerate}
        \item là encore, on peut placer des sous-listes ;
        \item ça marche exactement pareil que pour les listes non-numérotées ;
        \begin{itemize}
            \item on peut même mélanger les deux types de listes comme ici ;
            \item et ici.
        \end{itemize}
    \end{enumerate}
\end{enumerate}

\begin{table}[htb]
    % La commande '\centering' sert encore à centrer la table.
    \centering
    % Cette commande permet de colorer une ligne sur deux, pour améliorer la visibilité
    \rowcolors{1}{LightColor}{}
    % La commande '\tabular' créé la table proprement dite.
    % le paramètre liste les colonnes de la table. Chaque lettre représente une colonne
    % et son alignement : l=alignée à gauche, r=alignée à droite et c=centrée.
    % Ici, on a deux colonnes alignées à gauche.
    \begin{tabular}{l l}
        \hline
        % La commande '\rowcolor' colorie la ligne de titre de couleur plus foncée.
        \rowcolor{DarkColor}
        % Le contenu de la table est défini ligne à ligne. Les colonnes
        % sont séparées par le caractère '&' et la fin de la ligne est
        % indiquée par la chaine '\\' (obligatoire).
        % La ligne de titre doit etre en gras (commande '\textbf')
        \textbf{Type d'objet}   & \textbf{Préfixe}  \\
        % La commande '\hline' insère une séparation d'épaisseur normale.
        \hline
        section                 & \texttt{sec}  \\
        figure                  & \texttt{fig}  \\
        table                   & \texttt{tab}  \\
        equation                & \texttt{eq}   \\
        \hline
    \end{tabular}
    % La légende est définie comme pour une figure.
    \caption{Préfixes utilisés conventionnellement pour définir les noms des labels.}
    % Le label aussi est défini comme pour une figure, mais en utilisant le préfixe 'tab' plutot que 'fig'.
    \label{tab:prefixes}
\end{table}

\end{document}

